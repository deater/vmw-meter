\documentclass[11pt]{article}

\usepackage{url}
\usepackage{hyperref}

\begin{document}

\begin{center}
{\Large \bf Building the VMW Raspberry Pi / AY-3-8910 Chiptune Player}\\
\url{http://deater.net/weave/vmwprod/hardware/ay-3-8910/}\\
by Vincent M. Weaver\\
2 June 2017
\end{center}


\section{Introduction}

This is a work in progress.  I will update it as I complete more
of the project.



\section{Building}

\subsection{The Display}

\begin{table}

\caption{Display Parts.~\label{table:display_parts}}
\centering
\begin{tabular}{|c|c|c|c|c|c|}
\hline
NAME		& WHERE		& PART\#	& QTY	& PRICE		& TOTAL \\
\hline
\hline
PCB		& OSH-PARK	& VMW-DISPLAY	& 1	& \$40.00	& \$40.00 \\
\hline
i2c-14seg	& AdaFruit	& ?		& 3	&		&	\\
\hline
i2c-8x16	& AdaFruit	& ?		& 1	&		&	\\
\hline
ht16k33 breaout	& AdaFruit	& ?		& 1	&		&	\\
\hline
Buttons		& AdaFruit	& ?		& 8	&		&	\\
\hline
OWO Resistors	& ?		& 		& 8	& 		&	\\
\hline
Diode		& ?		& ?		& 1	& 		&	\\
\hline
RGY 10-seg LED	& Jameco	& ?		& 6	& 		&	\\
\hline
1/8" spacers	& Mouser	& ?		& ?	& 		&	\\
\hline
\#2 Screws	& Mouser	& ?		& ?	& 		&	\\
\hline
4-pin header	& ?		& ?		& ?	& 		&	\\
\hline
20-pin sockets	& ?		& ?		& 6	& 		&	\\
\hline
header		& Jameco	& 152734	& 2	& 		&	\\
.1" 4pin HSG	& 		&		&	&		&	\\
\hline
pins		& Jameco	& ????		& 8	&		&	\\
\hline
\hline
		&		&		&	&		&	\\
\hline
\end{tabular}
\end{table}

Build instructions for the i2c display board.

\begin{enumerate}
\item	Get the purple PCBs from OSH Park
\item	File down the little nubs around the edges, especially in the speaker
	areas
\item	Make an i2c cable.  see the pi cluster directions.
	not labeled on Mark1 PCB
	square is VDD, then it's SDA, GND, SCL

\item	First solder on the things on the back.  First the 4 pin socket.

\item	Next to the ht16k33 breakout.  (Build it first, putting the pin
	headers on each side)  We leave it at default i2c address (0x70)

\item	Now might be a good time to hook things up and test with i2c-scan
	to see if it is detecting properly.

\item	Next the diode (band direction is indicated)
	and 8 39k  resistors (orange white orange)

\item	Next solder in the 20-pin sockets

\item	Next the 8 switches.  There are bumps on the bottom of the
	switches that slot into holes on the board, which should
	ensure correct orientation.

\item	Now it's time for the adafruit 14-segment LED boards.
	First solder the headers onto them.

	Next, assuming you are using \#2 screws, we need to enlarge the
	drill holes on the Adafruit boards as they are just slightly too
	small.  I used a file, a drill can maybe be used.

	Solder the three boards to have different i2c addresses.
	Left is 101, middle 110, right 111.

	Now set up the spacers and screws.  This didn't really go according
	to plan (mostly due to nut size and the lower holes being in the
	wrong place on the Mark1 board).

	TODO: diagram

	Each board will be floating a bit in the air to avoid shorting out
	against component leads from the other side.  The height is equal
	to one 1/8" spacer plus a nut length.  

	On the top row, put the screw from the bottom, with a spacer and a nut.
	The board should slip onto this.  There is not enough clearance
	to put an additiona nut on top.
	On the lower ones, put the screw in from *the top* going through 
	a spacer with a nut on the other side.
	The bottom holes on the middle display are too close to the ones
	on either side, so leave those off.

\item	Now put on the Adafruit 8x16 LED display.  Attach the header.
	Configure for i2c address ???? (check on that)
	Expand all four holes on the board.
	The holes all work for this one, so put the screw from the bottom,
	with a spacer and nut on top.

\item	Put the 10-segment bargraph displays into the sockets.
	On the left side of the board the lettering on the displays should
	face down, on the right side they should face up.

\item	Put the five 0.5" standoffs into the board from the bottom, with
	a washer and nut on top.  The top center nut won't fit w/o shorting
	the switch, so leave it out for now.

\item	Put button tops onto the buttons (colors are up to you)

\item	Plug in the i2c wire and test (if you haven't been testing all along)
\end{enumerate}

Sound Board

	AY-3-8910	E-bay		?		2
	3.5mm Audio	DigiKey		CP1-3525NG-ND	1
		Stereo
	1MHz Osc	?		?		1
	Osc Socket					1
	74HC595N	?		?		2
	PCB		OSHPARK		VMW-SOUND	1	\$40.00
	40-pin header	AdaFruit
	Level Shifter	AdaFruit	?		2
	MAX98306 Break	AdaFruit	987		1
		3.7W Class D
	Resistors
	Capacitors
	1-wire DS1280	AdaFruit	?		1
	4-pin header
	40-pin sockets	Jameco		112311		2
	14-pin sockets	Jameco		37197		3
	14-pin osc	Jameco		133006		1
		socket
	16 pin socket	Jameco		37402		2
	8 pin socket					1
	0.1uF caps
	4 Ohm Speakers	AdaFruit	?		1
	spdt switch	Jameco		588220		1

	Optional:

	i2c EEPROM
	8-pin socket
	i2c-RTC

Directions

	1. Get the PCB
	2. Put the sockets on the AY side first?
		two 40-pin sockets
		oscillator socket
		the 14-pins which seem to be higher than 40-pin
	3. Solder on the audio jack
	5. solder on the two 1k and 10k resistors
	Flip sides
	solder on 2 16-pin sockets
	solder on 8-pin socket
	solder on 3 10k resistors, 2 1-k resistors, 2 3.9k



	flip back
	solder on the pi header.  attach to pi, then screw in place
		with standoffs. then solder
	solder on five 0.1uF caps
	solder on two 100uF caps make sure minus pin in minus hole
	4. Solder on the spdt switch (this is later, makes resistors
	on other side easier)


	flip back

	solder on 3 bypass caps
	DS18B20 (be sure to line up)
	two 4-pin connectors
	write protect 2-pin header
	RTC (if desired)

	flip back

	MAX
		easiest to start with fresh board.
		Only do the 9 wide header and take 4 extra pins
		and put then in the output holes.
		solder the board to these, then solder the whole thing
		to the board
	
	slide switch
		screw in place
		cut small 1/2" or so pieces of wire and solder in place
		two toward top side of main board are ground, 
		bottom two are signal

	Testing:
		5V+3.3V+GND continuity
		Hook up a pi.  Check for smoke.
		Run i2cdetect -y 1 and see if RTC found at 0x68
		Check that the 1-wire DS1820 shows up
			

	pput in chips

	snip the 1-pin indicator of the oscillator otherwise it's
	a tight fit

	wrap the standoff in electrical tape
		

Overall:
	Case		?		?		?
	XA Power Supply	?		?		?
	Mounting Screws
	Raspberry Pi2	?		?		?	\$35.00
	SD-Card		?		?		?


Building the case:

	Drill the holes and openings as indicated.

	Screw holes drilled with 7/64th bit

	On the pi, put two extra washers on the one side so that it lays
	flat.

	put electrical tape around the standofs for close clearance
	
	quarter inch holes and used coping saw to cut out openings





ERRATA!

the standoff behind the slider is too close to various components
the standoff on the bottom left completely blocks the memory card slot

mk1 board:
the capacitors should be at least 300 mil not 100 mil
oscillator too close to the chip
use curves not so many right angles
standoffs in bad places.


\end{document}

